\documentclass[a4papper, 14pt]{book}

%%%%%%%%%%%%%%%%%%%%%%%%%%%%%%%%%%%%%%% Работа с русским языком
\usepackage{cmap}					% поиск в PDF
\usepackage{mathtext} 				% русские буквы в формулах
\usepackage[T2A]{fontenc}			% кодировка
\usepackage[utf8]{inputenc}			% кодировка исходного текста
\usepackage[english,russian]{babel}	% локализация и переносы
%\parindent 0 cm 
\usepackage{indentfirst} % красная строка

%%%%%%%%%%%%%%%%%%%%%%%%%%%%%%%% Дополнительная работа с математикой
\usepackage{amsmath,amsfonts,amssymb,amsthm,mathtools} % AMS
\usepackage{icomma} % "Умная" запятая: $0,2$ --- число, $0, 2$ --- перечисление

%5%%%%%%%%%%%%%%%%%%%%%%%%%%%%%%%%%%% Номера формул
\mathtoolsset{showonlyrefs=true} % Показывать номера только у тех формул, на которые есть \eqref{} в тексте.
%\usepackage{leqno} % Нумерация формул слева

%%%%%%%%%%%%%%%%%%%%%%%%%%%%%%%%%%%%%% Свои команды
\DeclareMathOperator{\he}{\mathop{\text{\cyrtshe}}}
\renewcommand{\epsilon}{\ensuremath{\varepsilon}}
\renewcommand{\phi}{\ensuremath{\varphi}}
\renewcommand{\kappa}{\ensuremath{\varkappa}}
\renewcommand{\le}{\ensuremath{\leqslant}}
\renewcommand{\leq}{\ensuremath{\leqslant}}
\renewcommand{\geq}{\ensuremath{\geqslant}}
\renewcommand{\emptyset}{\ensuremath\varnothing}
\newcommand{\diad}{\ensuremath\otimes}
%\renewcommand{\int}{\int\limits}
\newcommand{\rx}[3]{\ensuremath\mf{r}_{#3}\left({\mathbf{#1}}_{#2} \right)}
\newcommand{\F}{\ensuremath\left(\mf{x}\right)}
\newcommand{\wx}[3]{\ensuremath\mf{w}_{#3}\left({\mathbf{#1}}_{#2} \right)}
\newcommand{\lf}{\ensuremath \left(}
\newcommand{\ri}{\ensuremath \right)}
\newcommand{\mt}{{\textbf{мт}}}
\newcommand{\mf}[1]{\ensuremath\mathbf{#1}}
%%%%%%%%%%%%%%%%%%%%%%%%% Перенос знаков в формулах (по Львовскому)
\newcommand*{\hm}[1]{#1\nobreak\discretionary{}
	{\hbox{$\mathsurround=0pt #1$}}{}}

%%%%%%%%%%%%%%%%%%%%%%%%%%%%%%%%%%%%%%%%%%%%%%%%%% Работа с картинками
\usepackage{graphicx}  % Для вставки рисунков
\graphicspath{{images/}{images/}}  % папки с картинками
\setlength\fboxsep{1.5pt} % Отступ рамки \fbox{} от рисунка
\setlength\fboxrule{1pt} % Толщина линий рамки \fbox{}
\usepackage{wrapfig} % Обтекание рисунков текстом
\usepackage{caption}
\usepackage{float}
\usepackage{floatflt}
\usepackage{listings}
\usepackage{wasysym}

%%%%%%%%%%%%%%%%%%%%%%%%%%%%%%% двоеточие на точку в подкиси к рисункам
\RequirePackage{caption}            
\DeclareCaptionLabelSeparator{defffis}{. }
\captionsetup{justification=centering,labelsep=defffis}

%%% %%%%%%%%%%%%%%%%%%%%%%%%%%%%%%%%%%%Работа с таблицами
\usepackage{array,tabularx,tabulary,booktabs} % Дополнительная работа с таблицами
\usepackage{longtable}  % Длинные таблицы
\usepackage{multirow} % Слияние строк в таблице

%%%%%%%%%%%%%%%%%%%%%%%%%%%%%%%%%%%%%%%%%%%% Теоремы
\theoremstyle{plain} % Это стиль по умолчанию, его можно не переопределять.
\newtheorem{theorem}{Теорема}[subsection]
\newtheorem{proposition}[theorem]{Утверждение}

\theoremstyle{definition} % "Определение"
\newtheorem{corollary}{Следствие}[theorem]
\newtheorem{problem}{Задача}[section]

\theoremstyle{remark} % "Примечание"
\newtheorem*{nonum}{Решение}

%%%%%%%%%%%%%%%%%%%%%%%%%%%%%%%%%%%%%%%%%%%%%%% Программирование
\usepackage{etoolbox} % логические операторы

%%%%%%%%%%%%%%%%%%%%%%%%%%%%%%%%%%%%%%%%%%%%%% Страница
\usepackage{extsizes} % Возможность сделать 14-й шрифт
%\usepackage{geometry} % Простой способ задавать поля
%	\geometry{top=10mm}
%	\geometry{bottom=20mm}
%	\geometry{left=10mm}
%	\geometry{right=10mm}
%	\geometry{bindingoffset=0cm}
%%%%
\usepackage{fancyhdr} % Колонтитулы
\pagestyle{myheadings}
\renewcommand{\headrulewidth}{0.95mm}  % Толщина линейки, отчеркивающей верхний колонтитул
\lfoot{} %Нижний левый
\rfoot{} % нижний правый
\rhead{} % верхний правый
\chead{} % верхний в центре
\lhead{} % Верхний левый
\cfoot{} %Нижний в центре По умолчанию здесь номер страницы

\usepackage{setspace} % Интерлиньяж
%\onehalfspacing % Интерлиньяж 1.5
%\doublespacing % Интерлиньяж 2
%\singlespacing % Интерлиньяж 1

\usepackage{lastpage} % Узнать, сколько всего страниц в документе.

\usepackage{soulutf8} % Модификаторы начертания

\usepackage{hyperref}
\usepackage[usenames,dvipsnames,svgnames,table,rgb]{xcolor}
\hypersetup{				% Гиперссылки
	unicode=true,           % русские буквы в раздела PDF
	pdftitle={Заголовок},   % Заголовок
	pdfauthor={Автор},      % Автор
	pdfsubject={Тема},      % Тема
	pdfcreator={Создатель}, % Создатель
	pdfproducer={Производитель}, % Производитель
	pdfkeywords={keyword1} {key2} {key3}, % Ключевые слова
	colorlinks=true,       	% false: ссылки в рамках; true: цветные ссылки
	linkcolor=red,          % внутренние ссылки
	citecolor=green,        % на библиографию
	filecolor=magenta,      % на файлы
	urlcolor=cyan           % на URL
}

%\renewcommand{\familydefault}{\sfdefault} % Начертание шрифта

\usepackage{multicol} % Несколько колонок

\usepackage{csquotes} % Еще инструменты для ссылок

\usepackage[backend=biber,bibencoding=utf8,sorting=ynt,maxcitenames=2,style=authoryear]{biblatex}
\addbibresource{bib1.bib}

\usepackage{tikz} % Работа с графикой
\usepackage{pgfplots}
\usepackage{pgfplotstable}



%\author{Рыжак Евгений Измаилович}
%\title{Лекции по механике сплошных сред (механика твердого и деформируемого тела)}

\date{\today}
%\creator{Каныгин Юрий Юрьевич и Грибова Ольга Борисовна, 432 группа}
\pgfplotsset{compat=1.14}





\begin{document} % конец преамбулы, начало документа
	\thispagestyle{empty}
	\begin{center}
		\textit{Федеральное государственное автономное учреждение \\
			высшего профессионального образования}
		\vspace{0.5ex}
		
		\textbf{ <<МОСКОВСКИЙ ФИЗИКО-ТЕХНИЧЕСКИЙ ИНСТИТУТ (ГОСУДАРСТВЕННЫЙ УНИВЕРСИТЕТ)>>}
	\end{center}
	\vspace{13ex}
	\begin{flushright}
		\noindent
		%\textit{Рыжак Евгений Измаилович}
		\\
		%\textit{студент факультета экономики \\(группа 211И)}
	\end{flushright}
	\begin{center}
		\vspace{13ex}
		\so{\textbf{Конспект лекций}}
		\vspace{1ex}
		
		по механике сплошных сред 
		
		
		%	на тему:
		
		%	\textbf{\textit{<<Заголовок>>}}
		
		\vfill
		Москва 2017
	\end{center}
	%\maketitle
	\newpage
	%\setcounter{part}{1}
	%\part{}
	{\Large\textbf{Предисловие}}\\
	
	
	\newpage
	\tableofcontents
	\newpage
	\def\thesection{Введение}
	\section{}
	Постулируется непрерывность преобразования из одного состояния в другое. И не может быть так, что часть конечного объема после трансформации стала бесконечной. Если есть последовательность состояний тела и эта последовательность имела предел, то так будет и в любом другом состоянии системы. Это и есть сплошность, фактически, непрерывность. \\
	
	Итак, принципиальное различие механики сплошных сред (МСС) от механики дискретных сред (МДС) состоит в том, что вещество в МДС представляется в виде набора материальных точек в конкретных точках пространства, в то время как в МСС материальные точки полностью заполняют предоставленный объем: свойства задаются не для конкретной точки, а для набора, т.е. для некоторого объема. \\
	
	Как и в механике дискретных сред, масса тела неизменна, положительна, если объем положителен, и аддитивна т.е. если части тела не пересекаются или пересекаются по множеству нулевой меры, то масса всех этих частей равна сумме масс этих частей в отдельности. \\
	
	Важное понятие, используемое нами в дальнейшем "--- отсчётное описание. Это способ идентификации материальных точек (в дальнейшем \mt).
	Среди всех состояний (конфигураций) тела в какой-то момент времени $t$ существует одно, котрое мы мысленно фиксируем. Эту конфигурацию будем называть \textbf{отсчётной} и обозначать $\kappa$, она неизменна. Остальные же конфигурации зависят от времени, поэтому {\textbf{актуальную}} конфигурацию (в данный момент времени) обозначим  $\chi\left(t\right)$. Местоположение мт в отсчётной конфигурации служит <<именем>> этой мт. \\
	
	Выбирается некоторое состояние тела  (отсчётная конфигурация) в какой-то момент времени $t$, именами точек служит радиус-векторы. Этот метод похож на учет населения с помощью постоянной прописки. Для того, чтобы узнать текущую (актуальную) конфигурацию, для обозначения которой мы будем использовать $\chi(t)$ мы должны задаться отображением:
	
	\begin{equation}\label{eq:transform}
	\mf{r} = \mf{r}\lf\mf{x},t\ri
	\end{equation}
	
	Движение материальной точки --- это отображение $\mf{r}$ при фиксированном $\mf{x}$. В дальнейшем предполагается, что $\mf{r}\in C^2$, или хотя бы кусочно-непрерывно дифференцируемо по $\mf{x}$. И действительно, рассмотрим стержень, одну часть которого растянем, а другую сожмем Градиент тут будет разрывен, по-этому естественно, что в МСС присутствуют разрывы производных. 
	
	\setcounter{chapter}{1}
	\setcounter{section}{0}
	\def\thesection{Лекция \arabic{section}}
	\section{}
	
	Введем градиент $\mf{r}$ по $\mf{x}$:
	
	\begin{equation}\label{eq:grad}
	F \lf \mf{x}, t\ri \equiv \nabla_{\kappa}\diad\mf{r}\lf(\mf{x}, t\ri)
	\end{equation}
	
	Индекс при векторе набла указывает, что дифференцирование идет по $\mf{x}$ т.~е. по радиусу вектору отсчётной конфигурации. Что означает на самом деле формула \eqref{eq:grad}? Вспомнив определение дифференциала, не трудно понять, что $F$ --- это линейный оператор, отображающий $d\mf{x}$ в $d\mf{r}$. 
	
	\begin{equation}
	d\mf{r} = d\mf{x}\cdot F
	\end{equation}
	
	В этом градиенте содержится все, что происходит с окрестностью точки. Договоримся, что $\det F$ "--- коэффициент преобразования объема. В дальнейшем будем считать, что $\det F\neq 0$, иными словами:
	
	\begin{equation}
	d\mf{x} = d\mf{r}\cdot F^{-1}
	\end{equation}
	
	Раз определитель отличен от 0 и по нашему договору все меняется непрерывно, то детерминант не может менять знак ( иначе он обратился бы в 0 в какой-то момент). Какой же он, больше или меньше нуля? Если $\mf{r}_{\text{отс}} = \mf{x} \Rightarrow F\equiv I$, $\det I = 1 >0$.\\
	
	Мы всегда будем полагать, что определитель больше нуля, тогда это к тому же и коэффициент преобразования объема:
	
	\begin{equation}
	dV = \left(\det F\right) dV_{\kappa}
	\end{equation}
	
	Предположения о массе сделанные выше влекут за собой соображения о плотности массы. Масса есть интеграл этой плотности по окрестности $dm = \rho_{\kappa}dV_{\kappa} = \rho dV = \rho\, \det F\, dV_{\kappa}$ :
	\begin{equation}\label{eq:plotnost}
	\boxed{\rho_{\kappa}\left(\mf{x}\right) = \rho\left(\mf{x}, t\right)  \det F\left(\mf{x}, t\right)} 
	\end{equation}
	
	Есть два типа описания физических величин:
	
	\begin{center}\begin{tabular}{p{8cm}p{8cm}}
			Отсчётное описание физической величины $\mf{r} = \mf{r}\left(\mf{x}, t\right)$: $$\Psi = \Psi\left(\mf{x}, t\right)$$ &  Пространственное описание, выражающееся формулами от $t$ и $\mf{r}$, чаще всего такое описание используется в гидромеханике. Мы тоже будем иногда прибегать к такому способу. Для этого вводится специальная функция $$\Psi = \Psi\left(\mf{r}, t\right)$$ \\ 
		\end{tabular}
	\end{center}
	
	Производную по времени будем называть материальной производной:
	
	$$
	\dot{\Psi} = \left( \dfrac{\partial\Psi}{\partial t} \right)_{\mf{x}}
	$$
	
	Пространственной производной будем называть величину:
	
	$$
	\left(\dfrac{\partial\Psi}{\partial t}\right)_{\mf{r}}
	$$
	
	Чтобы понять в чем разница, можно привести пример с рекой и температурой. Если ставить термометр на якорь, то это все равно, что пространственная производная, то есть фиксировано положение термометра в реке и измеряется температура конкретной точки пространства. Если термометр пустить по течению, то будет измерятся температура конкретной точки реки, это соответствует временной производной. Заметим, что $\dot{\mf{r}} = \mf{v}\left(\mf{x}, t\right)$. Найдем связь между временной и пространственной производной (формула Эйлера):
	
	\begin{equation}
	\dot{\Psi} \equiv \left(\dfrac{\partial\Psi}{\partial t}\right)_{\mf{x}} = \left(\dfrac{\partial\Psi}{\partial t}\right)_{\mf{r}} +	\mf{v}\cdot\nabla\diad\Psi
	\end{equation}
	Формула получается применением дифференцирования сложной функции, если подставить в выражение $\Psi = \Psi\left(\mf{x}, t\right)$ формулу \eqref{eq:transform}.
	$$
	\Psi = \Psi\left(\mf{r}\left(\mf{x}, t\right), t\right)
	$$
	В частности:
	$$
	\dot{\mf{v}} = \left(\dfrac{\partial \mf{v}}{\partial t}\right)_{\mf{r}} + \mf{v}\cdot\nabla\diad\mf{v}
	$$
	В стационарном течении получаем $\dot{\mf{v}}=\mf{v}\cdot\nabla\diad\mf{v}$, так же можно рассмотреть сдвиговое течение (ярким примером которого может служить скосившаяся стопка бумаг или колода карт), тогда $\mf{v}\cdot\nabla\diad\mf{v}$
	Найдем связь между градиентами в отсчётной и актуальной конфигурациях:
	$$
	\left(d\Psi\right)_t = d\mf{x}\cdot\nabla_{\kappa}\diad\Psi=\mf{r}\cdot\nabla\diad\Psi\stackrel{\eqref{eq:transform}}=d\mf{x}\cdot	F\cdot\nabla\diad\Psi
	$$
	\begin{equation}\label{eq: gradients}
	\nabla_{\kappa} = F\cdot\nabla
	\end{equation}
	Вспомним о независимости порядка дифференцирования независимых переменных и получим выражение, которое понадобится нам для вывода закона сохранения массы (ЗСМ):
	\begin{equation}\label{eq: proivod F}
	\dot{F}\left(\mf{x}, t\right) =\left(\nabla_{\kappa}\diad\mf{r}\left(\mf{x},	t\right)\right)^{\displaystyle .} = 		\nabla_{\kappa}\diad\dot{\mf{r}}\left(\mf{x}, t\right) = \nabla_{\kappa}\diad 	\mf{v}\left(\mf{x}, t\right) 
	\end{equation}
	Также, перед выводом ЗСМ полезно вспомнить следующую формулу:
	\begin{equation}\label{eq:proizvd det}
	\left(\det F\right)^{\displaystyle .} = \dot{F}:\det F\left(F^{-1}\right)^T
	\end{equation}
	Теперь у нас есть все, для вывода ЗСМ. Продифференцируем выражение \eqref{eq:plotnost} :
	$$
	\dot{\rho}\det F + \rho\left(\det F\right)^{\displaystyle .} = 0
	$$
	Вспомнив выражение \eqref{eq:proizvd det} получаем:
	$$
	\dot{\rho}\det F + \rho\left(\det F\right)\dot{F}:\left(F^{-1}\right)^T = 0
	$$
	Так как мы обговаривали что определитель считаем положительным, то на него можно поделить, так же вспомнив про выражение \eqref{eq: proivod F} приходим к равенству:
	$$
	\dot{\rho} + \rho\nabla_{\kappa}\diad\dot{\mf{v}}:\left(F^{-1}\right)^T = 0
	$$
	Из тензорной алгебры известно соотношение $A:B = I:\left(A\cdot B^T \right) =I:\left(B^T \cdot A\right)$, тогда наше выражение примет вид:
	$$
	\dot{\rho} + \rho I : F^{-1}\cdot\nabla_{\kappa}\diad\mf{v} = 0
	$$
	Если вспомнить соотношение между градиентами (формула \eqref{eq: gradients}) и свойства двойного скалярного произведения, мы получим окончательно:
	\begin{equation}
	\left(\dfrac{\partial\rho}{\partial t}\right)_{\mf{r}} + \mf{v}\cdot\nabla\rho + 	\rho\nabla\cdot\mf{v} = 0
	\end{equation}
	Займемся выводом еще одной интересной формулы (скорость изменения объема) $\left(\det F \right)^{\displaystyle .} = \left(\det F \right)\nabla\cdot \mf{v}$. Пусть тело занимает в актуальной конфигурации область $B$. 
	\begin{equation}
	V\left(t\right) = \int\limits_B dV = \int\limits_{B_{\kappa}} \left(\det F \right) dV_{\kappa}
	\end{equation}
	\begin{multline}
	\dot{V}\left(t\right) = \left(\int\limits_B dV\right)^{\displaystyle\cdot} = \left(\,\int\limits_{B_{\kappa}} \left(\det F \right) dV_{\kappa}\right)^{\displaystyle\cdot } =\\
	= \int\limits_{B_{\kappa}} \left(\det F \right)^{\displaystyle\cdot } dV_{\kappa} = \int\limits_{B_{\kappa}} \left(\det F\, \nabla\cdot\mf{v}\right) \dfrac{dV_{\kappa}}{dV} =\\ 
	= \int\limits_B \nabla\cdot\mf{v} dV = \int\limits_{\partial B} \left(\mf{n}\cdot\mf{v}\right) d\Sigma
	\end{multline}
	\def\thesubsection{\arabic{section}.\arabic{subsection}}
	\subsection{Теория конечных деформаций}
	Вспомним теорему Коши о полярном разложении
	\begin{equation}
	F = U\cdot R = R\cdot U',
	\end{equation}
	где $U$~---~симметричный положительно определенный тензор, а $R$~---~ортогональный. Если представить $U$ в главных осях, то его детерминант будет произведением собственных чисел ($\det U = u_1 u_2 u_3$):
	\begin{equation}
	U = u_1\mf{e}_1\diad\mf{e}_1 + u_2\mf{e}_2\diad\mf{e}_2 + u_3\mf{e}_3\diad\mf{e}_3 
	\end{equation}
	Так как $U$ положительно определенный, то и его собственные числа положительны, тогда:
	\begin{equation}
	\det F = \det U\det R > 0\Rightarrow\det R >0 \Rightarrow \det R = 1 \hbox{ т.к. $R$ ортоганален.}
	\end{equation}   
	Вспомним как связаны между собой $U'$ и $U$:
	\begin{multline}
	U' = R^T U R = U*R\Rightarrow\\
	\Rightarrow U' = u_1 \mf{e}_1^{\prime} \diad\mf{e}_1^{\prime} + u_2 \mf{e}_2^{\prime} \diad\mf{e}_2^{\prime} + u_3 \mf{e}_3^{\prime} \diad\mf{e}_3^{\prime}, 
	\end{multline}
	где $ \mf{e}_i^{\prime} = \mf{e}_i \cdot R$. Вспоминая формулу $\eqref{eq:transform}$ можем получить: $d\mf{r} = d\mf{x}\cdot F = d\mf{x}\cdot UR = d\mf{x}\cdot RU' $. Тогда можно дать следующую физическую интерпретацию. $U$~---~левый тензор чистого растяжения (растяжение происходит вдоль осей $\mf{e}_i$~---~осей растяжения в актуальной конфигурации, с коэффициентом $u_i$), $R$~--- ~тензор поворота. $U'$~---~правый тензор чистого растяжения (растяжение происходит вдоль осей $\mf{e}_i^{\prime}$~---~осей растяжения в отсчётной конфигурации, с коэффициентом $u_i$).\\
	
	Пусть $d\mf{x} = \vert d\mf{x}\vert\mf{e}_i$, тогда:
	\begin{equation}
	d\mf{r} = \vert d\mf{x}\vert \mf{e}_i \cdot F = \vert d\mf{x}\vert \cdot UR 
	=\vert d\mf{x}\vert u_i \mf{e}_i\cdot R = \vert d\mf{x}\vert u_i^{\prime} \mf{e}_i
	\end{equation}
	А что если $d\mf{x}\nparallel\mf{e}_i$? Какой будет коэффициент растяжения? Введем два тензора $F\cdot F^T = U\cdot R \cdot R^T \cdot U = U^2$~---~ левый тензор Коши-Грина. Аналогично $F\cdot F^T = U^{\prime 2}$~---~правый тензор Коши-Грина. Возьмем направленный вектор $d\mf{x} = \vert d\mf{x}\vert\mf{e}$. 
	\begin{equation}
	\vert d\mf{r}\vert^2 = d\mf{r}\cdot d\mf{r} = d\mf{x}\cdot F\cdot d\mf{x}\cdot F  = d\mf{x}\cdot F\cdot F^T \cdot d\mf{x} = d\mf{x}\cdot U^2 \cdot d\mf{x} = \vert d\mf{x}\vert^2 \mf{e}\cdot U^2 \cdot \mf{e}
	\end{equation}
	\begin{equation}
	\dfrac{\vert d\mf{r}\vert}{\vert d\mf{x}\vert} = \sqrt{\mf{e}\cdot F\cdot F^T\cdot \mf{e}}
	\end{equation}
	Введем понятие угла сдвига. Возьмем в отсчётной конфигурации два взаимно-ортогональных элемента $d\mf{x}$ и $d\mf{x}{\,^\prime}$, тогда:
	\begin{equation}
	d\mf{r}\cdot d\mf{r}^{\prime} = \vert d\mf{r}\vert \vert d\mf{r}^{\prime}\vert\cos\alpha= \vert d\mf{r}\vert \vert d\mf{r}^{\prime}\vert\sin\gamma
	\end{equation}
	С другой стороны:
	\begin{multline}
	d\mf{r}\cdot d\mf{r}^{\prime} = d\mf{x}\cdot F\cdot d\mf{x}^{\prime}\cdot F  = d\mf{x}\cdot F\cdot F^T \cdot d\mf{x}^{\prime} =\\
	= \vert d\mf{x}\vert\vert d\mf{x}^{\prime}\vert\mf{e}\cdot F\cdot F^T\cdot \mf{e}^{\prime} = \vert d\mf{r}\vert \vert d\mf{r}^{\prime}\vert\sin\gamma
	\end{multline}
	Таким образом получаем выражение для синуса угла сдвига:
	\begin{equation}
	\sin\gamma = \dfrac{\mf{e}\cdot F \cdot F^T \cdot \mf{e}^{\prime}}{\vert d\mf{r} \vert\vert d\mf{r}{ \,^\prime}\vert} \cdot \vert d\mf{x}\vert\vert d\mf{x}^{\prime}\vert =  \dfrac{\mf{e}\cdot F \cdot F^T \cdot \mf{e}^{\prime}}{\dfrac{\vert d\mf{r} \vert}{\vert d\mf{x}\vert}\dfrac{\vert d\mf{r}^{\prime}\vert}{\vert d\mf{x}^{\prime}\vert}}
	\end{equation}
	\begin{equation}\label{eq:singamma}
	\sin\gamma = \dfrac{\mf{e}\cdot F \cdot F^T \cdot\mf{e}^{\prime}}{\sqrt{\mf{e}\cdot F\cdot F^T\cdot \mf{e}}\,\sqrt{\mf{e}{^\prime}\cdot F\cdot F^T\cdot \mf{e}}{^\prime}}
	\end{equation}
	Если $\mf{e} = \mf{e}_1,\,\mf{e}{^\prime}=\mf{e}_2 $, тогда в выражении \eqref{eq:singamma} будет стоять 0. Таким образом, главные векторы только поворачиваются, а углы между ними не меняются. \\
	
	Получим тензор конечных деформаций. Для этого введем понятие вектора смещения $\mf{w}\left(\mf{x}\right)  = \mf{r}\left(\mf{x}\right) - \mf{x}$. Выразим из него вектор $\mf{r}$ и подставим в \eqref{eq:grad}. 
	\begin{align}
	&F = I +\nabla_{\kappa}\diad \mf{w},&   F^T = I + \left(\nabla_{\kappa}\diad \mf{w}\right)^T
	\end{align}
	\begin{equation}
	F\cdot F^T = I + \nabla_{\kappa}\diad \mf{w} +\left(\nabla_{\kappa}\diad \mf{w}\right)^T + \nabla_{\kappa}\diad \mf{w}\cdot\left(\nabla_{\kappa}\diad \mf{w}\right)^T
	\end{equation}
	И тензором конечных деформаций назовем величину: 
	\begin{equation}
	\dfrac{F\cdot F^T - I}{2} = \dfrac{1}{2}\left(\nabla_{\kappa}\diad \mf{w} +\left(\nabla_{\kappa}\diad \mf{w}\right)^T + \nabla_{\kappa}\diad \mf{w}\cdot\left(\nabla_{\kappa}\diad \mf{w}\right)^T\right)
	\end{equation}
	\setcounter{chapter}{2}
	\section{}
	Продолжим тему предыдущей лекции. $\rx{x}{}{} = \mf{x}_0 + \left(\mf{x} - \mf{x}_0 \right)R_0 + \mf{w}_0$, где тензор $R_0$ "--- тензор поворота(без параллельного переноса). Что происходит с точкой $\mf{x}_0$? Если $\mf{x} = \mf{x}_0$, то $\rx{x}{0} = \mf{x}_0 + \mf{w}_0 = \mf{r}_0 \left(\mf{x}_0 \right)$. Получим поле смещений:
	\begin{equation}
	\wx{x}{}{} = \rx{x}{}{} - \mf{x} = \left(\mf{x} - \mf{x}_0\right)\cdot \left(R_0 - I\right) +\mf{w}_0
	\end{equation}
	\begin{equation}
	\nabla_{\kappa}\diad\wx{x}{}{} = R_0 - I
	\end{equation}
	Будем называть трансформацию однородной, если ее градиент постоянен. Заменим поворот тензором $F_0$, в смысл которого вложим и поворот и растяжение Для этого тензора справедливо полярное разложение Коши.
	\begin{equation}
	\rx{x}{}{} =\mf{x}_0 + \left( \mf{x} - \mf{x}_0 \right)F_0 + \mf{w}_0 = \mf{r}_0 + \left( \mf{x} - \mf{x}_0 \right)F_0.
	\end{equation}
	Рассмотрим два вида деформаций. 
	\subsection{Простой сдвиг}
	Для лучшего понимания и представление возьмем стопку бумаг, которую мы однородно перекосим. Это и будет простым сдвигом. Будем считать, что параллельный перенос $\mf{w}_0 = 0$, он не интересен. Обозначим за направление смещения единичный вектор $\mf{m}$, который будет перпендикулярный нормали к поверхности бумаги $\mf{n}$. Тогда запишем смещение:
	\begin{align}
	\wx{x}{}{} = \left[ \lf\mf{x} - \mf{x}_0\ri\cdot\mf{n}\right] \mf{m}\gamma &= \lf \mf{x} - \mf{x}_0\ri\cdot\lf\gamma\mf{n}\diad \mf{m}\ri \\
	\nabla_{\kappa}\diad\mf{w} = \gamma\mf{n}\diad\mf{m} &= \mf{n}\diad\lf\gamma\mf{m}\ri,
	\end{align}
	где вектор $\gamma\mf{m}$ "--- вектор вдоль которого смещается плоскость, находящаяся на единичном расстоянии от опорной плоскости. Тензор сдвига будет соответственно выглядеть:
	\begin{equation}\label{eq:sdvig}
	F = I + \gamma\mf{n}\diad\mf{n}
	\end{equation}
	\subsection{Растяжение и сжатие в некотором направлении}
	Будем пользоваться той же моделью (стопкой бумаг). Введем базис $\{\mf{e}_i\}$ так, чтобы третья ось была со-направленна с нормалью к поверхности, тогда
	\begin{equation}
	\rx{x}{}{} = \mf{x}_0 + \lf \mf{x} - \mf{x}_0 \ri \cdot\lf \mf{e}_1 \diad\mf{e}_1 + \mf{e}_2 \diad\mf{e}_2 + \lf 1+\epsilon\ri\mf{n}\diad\mf{n}\ri = I + \epsilon\mf{n}\diad\mf{n}.
	\end{equation}
	Последнее выражение описывает трансформацию растяжения или сжатия в направлении вектора нормали на величину $\epsilon$. Тогда поле смещений:
	\begin{align}
	\wx{x}{}{} = \lf\mf{x} - \mf{x}_0\ri \epsilon\mf{n}\diad\mf{n} &= \lf\mf{x} - \mf{x}_0\ri\lf F_0 - I\ri =  \lf\mf{x} - \mf{x}_0\ri \mf{n}\diad\epsilon\mf{n}.\\
	\nabla_{\kappa}\diad\wx{x}{}{}& = \mf{n}\diad\lf\epsilon\mf{n}\ri.
	\end{align}
	Таким образом тензор растяжений выглядит следующим образом:
	\begin{equation}\label{eq:pastizenie}
	F = I + \epsilon\mf{n}\diad\mf{n}.
	\end{equation}
	Когда градиент поля смещений диада, как в этом случае, то такую трансформацию мы назовем диадной. Вектор $\epsilon\mf{n}$ "--- вектор смещения плоскости, находящейся на единичном расстоянии от опорной плоскости на коэффициент \epsilon. \\
	Если трансформации осуществлять последовательно, то соответствующие им тензоры будут перемножаться. Это легко доказать, $d\mf{x}\cdot F = d\mf{r}$, $d\mf{r}\cdot F' = d\mf{r} { \,^\prime} = d\mf{x}\cdot F\cdot F'$. Если теперь перемножить \eqref{eq:sdvig} и \eqref{eq:pastizenie} (сдвиг на растяжение), то мы получим комбинацию (тензор, который отвечает и за сдвиг и за растяжение). Будем считать что мы работаем с малыми деформациями и отбрасывать члены, содержащие произведение коэффициентов сдвига и растяжения $\gamma$ и \epsilon. 
	\begin{multline}
	F = \lf I + \gamma\left[\mf{n}\diad\mf{m}\right] \ri\cdot\lf I + \epsilon\left[\mf{n}\diad\mf{n}\right] \ri=\\ = I + \epsilon
	\mf{n}\diad\mf{n} + \gamma\mf{n}\diad\mf{m} + 0 =\\
	= I +\mf{n}\diad\lf\epsilon\mf{n} + \gamma\mf{m}\ri
	\end{multline}
	\begin{equation}
	\nabla_{\kappa}\diad\mf{w} = \mf{n}\diad\lf\gamma\mf{m}+\epsilon\mf{n}\ri.
	\end{equation}
	Получили комбинацию сдвига и деформации, которая так же является диадной трансформацией.
	Чем более сложные примеры мы будем выбирать, тем проще будут формулы.\\
	Будем теперь продолжать мысль со стопкой бумаги, но, если раньше мы с ней аккуратно обращались (сдвигали, утолщали или сжимали) какими-то упорядоченными однородными движениями, то теперь произведем над стопкой беспорядочные действия: в одном месте сдвинем в одну сторону, в другом в другую, в третьем в третью, где-то сожмем, где-то утолщим... Тогда с этой системой опорных плоскостей произойдет какая-то линейная комбинация движений, но в каждом тонком слое своя. А все это записывается такой формулой:
	\begin{equation}
	\wx{x}{}{} = \mf{f}\lf\lf\mf{x} - \mf{x}_0\ri\cdot\mf{n}\ri. 
	\end{equation}
	То есть все зависит от аргумента $\lf\lf\mf{x} - \mf{x}_0\ri\cdot\mf{n}\ri$, который можно назвать $z$. Этот $z$ задает ту или иную плоскость в системе опорных плоскостей. И для каждой плоскости будет свой вектор смещения. Давайте убедимся в том, что локально это есть комбинация сдвига и растяжения. Найдем $\nabla_{\kappa}\diad\mf{w}$: \\
	\begin{equation}
	\wx{x}{}{} = dz \cdot \mf{f}{^\prime}\left(z\right) = \left(d\mf{x} \cdot \mf{n}\right) \cdot \mf{f}{^\prime} \left(\mf{x} \cdot \mf{n}\right) = d\mf{x} \cdot \mf{n} \diad \mf{f}{^\prime} (\mf{x} \cdot \mf{n})
	\end{equation}
	\begin{equation}
	\nabla_{\kappa}\diad\mf{w} = \mf{n} \diad \mf{f}{^\prime} \left(\mf{x} \cdot \mf{n}\right) = \mf{n} \diad \lf \gamma (z) \mf{m} (z)  + \epsilon (z) \mf{n} \ri.
	\end{equation}
	
	Усложним задачу еще: возьмем другую гладкую поверхность (не являющуюся плоскостью), каждый слой которой так же будет иметь свое смещение. Например, для сферы: $\lf\mf{x} - \mf{x}_0\ri\cdot\lf\mf{x} - \mf{x}_0\ri = a^2$. Теперь, если мы будем менять $a$, получим систему концентрических сфер с центром в $x_0$. На самом деле, любая система поверхностей можеть быть задана аналогичным образом, а именно: $\phi\lf\mf{x}\ri = \alpha$. И при изменении $\alpha$ получим систему непересекающихся поверхностей, каждая из которых задается своим значением $\alpha$. А теперь для этого случая зададим поле смещения примерно так же, как и раньше, только вместо $\lf \mf{x} \cdot \mf{n} \ri$ зададим $\phi \lf \mf{x} \ri$. Тогда опишем трансформацию:
	\begin{align}
	\wx{x}{}{} &= \mf{f} \lf \phi \lf \mf{x} \ri \ri \\ 
	\nabla_{\kappa} \diad \wx{x}{ }{ } &= \nabla_{\kappa} \phi \lf \mf{x} \ri \diad \mf{f} {\,^\prime} \lf \phi\lf \mf{x} \ri\ri \\
	\nabla_{\kappa} \phi\lf\mf{x}\ri &= \mf{n}\lf\mf{x}\ri\vert\nabla_{\kappa} \phi\nabla_{\kappa} \diad \wx{x}{ }{ } \vert \\
	\nabla_{\kappa} \diad \wx{x}{ }{ } &= \mf{n}\lf\mf{x}\ri\diad\{ \vert\nabla_{\kappa}\phi\lf\mf{x}\ri\vert\mf{f}{^\prime} \} \\
	\nabla_{\kappa} \diad \wx{x}{ }{ } &=\mf{n}\lf\mf{x}\ri\diad\lf \gamma\lf\mf{x}\ri\mf{m}\lf\mf{x}\ri + \epsilon\lf\mf{x}\ri\mf{n}\lf\mf{x}\ri \ri.\\
	\end{align}
	Последнее выражение показывает, что каждый тонкий слой, прилегающий к поверхности, претерпевает комбинацию сдвигов и растяжений. 	\subsection{Малые деформации и скоростные деформационные величины}
	Перейдем к другому вопросу, связанному с малыми деформациями и скоростными деформационными величинами, которые характеризуют то, что происходит в окрестности данного состояния. Представим, что есть некоторое состояние и некотрое движение, в процессе которого это состояние изменяется мало. Во многих случаях интересно именно то, как этот процесс происходит. Для рассмотрения этого строится некая интересная теория, которую мы с вами сейчас и рассмотрим. \\ 
	
	Прежде всего введем относительный градиент деформации. Сначала введем его формально, а потом разберем геометрический смысл. Все, что сейчас будет рассматриваться, будет относиться к точке с одинаковым $\mf{x}$, но к разным значениям $t$, поэтому аргумент $\mf{x}$ будем опускать
	\begin{align}
	F\left(t\right)  \equiv F\lf t,\, \mf{x}\ri
	\end{align}
	Зафиксируем состояние в некоторый, вообще говоря, любой, момент $t_0$. Относительным градиентом трансформации будем называть величину
	\begin{equation}
	F_{t_0}\lf t \ri = {\lf F\lf t_0 \ri\ri}^{-1} \cdot F\lf t \ri 
	\end{equation}
	Сразу очевидно, что при $t = t_0$ $F_{t_o} \lf t_0 \ri = I$. Фактически это градиент трансформации относительно новой конфигурации, которая совпадает с актуальной в момент времени $t_0$.
	\begin{align}
	&d\mf{r}\lf t\ri = d\mf{x} \cdot F\lf t \ri\\
	&d\mf{r}\lf t_0\ri = d\mf{x} \cdot F\lf t_0 \ri\\
	&d\mf{x} = d\mf{r}\lf t_0\ri \cdot {(F\lf t_0\ri)}^{-1}
	\end{align}
	\begin{equation}
	d\mf{r}\lf t\ri = d\mf{r}\lf t_0\ri \cdot {(F\lf t_0\ri)}^{-1} \cdot F\lf t \ri = d\mf{r}\lf t_0\ri \cdot F_{t_0}\lf t\ri
	\end{equation}
	Проделаем теперь некоторые манипуляции, подразумевая в дальнейшем под $F_{t_0}\lf t\ri$ тензор (относительный градиент трансформации), отображающий $d\mf{r}\lf t_0\ri$ в $d\mf{r}\lf t\ri$. Чтобы был более понятен смысл, давайте продифференцируем полученную ранее формулу
	\begin{equation}
	d\dot{\mf{r}}\lf t\ri = d\mf{r}\lf t_0\ri \cdot \dot{F}_{t_0}\lf t\ri\\
	\end{equation}
	Получили скорость изменения элемента $d\mf{r}$ и от чего она зависит. Сразу напрашивается вопрос, что же будет в момент времени $t_0$. Эта формула будет так же справедлива
	\begin{equation}\label{eq: dr/dt t=t0}
	d\dot{\mf{r}}\lf t\ri\vert_{t = t_0} = d\mf{r}\lf t_0\ri \cdot \dot{F}_{t_0}\lf t\ri\vert_{t = t_0}
	\end{equation}
	Эти манипуляции можно определить как случай, когда актуальная конфигурация берется за начало отсчета.\\
	Теперь давайте проделаем некоторые вычисления, связанные с этим дифференцированием. Сначала просто продифференцируем $F_{t_0}(t)$, а потом положим $t=t_0$. Но прежде чем мы начнем это делать, мы запишем некоторые формулы, а именно: для относительного градиента, который является невырожденным тензором с положительным детерминантом, мы можем записать полярное разложение Коши таким образом
	\begin{equation}
	F_{t_0} = U_{t_0}\lf t \ri\cdot R_{t_0}\lf t \ri
	\end{equation}
	Тензор $U$ является симметричным положительно определенным, а тензор $R$ является тензором ортогональным. Но если помнить, что в момент времени $t = t_0$,  $F_{t_o} \lf t_0 \ri = I$, то для $I$ полярное разложение имеет вид $I \cdot I$, то есть 
	\begin{equation}\label{local1}
	U_{t_0}\lf t_0 \ri = I ; R_{t_0}\lf t_0 \ri = I
	\end{equation}
	\begin{equation}
	F_{t_0}\lf t_0 \ri = I =U_{t_0}\lf t_0 \ri\cdot R_{t_0}\lf t_0 \ri =I\cdot I
	\end{equation}
	Теперь давайте дифференцировать, но пока что при произвольном $t$
	\begin{equation}
	\dot{F}_{t_0}\lf t \ri = \dot{U}_{t_0}\lf t \ri\cdot R_{t_0}\lf t \ri + U_{t_0}\lf t \ri\cdot \dot{R}_{t_0}\lf t \ri 
	\end{equation}
	С учетом \eqref{local1} в момент времени $t_0$
	\begin{equation}
	\dot{F}_{t_0}\lf t \ri\vert_{t = t_0} = \dot{U}_{t_0}\lf t \ri\vert_{t = t_0} + \dot{R}_{t_0}\lf t \ri\vert_{t = t_0}     
	\end{equation}
	$U_{t_0}$ при любом $t$ является симметричным тензором, значит, его производная по скалярному аргументу является им тоже. Его мы будем обозначать $\dot{\epsilon}\lf t_0 \ri $. Он характеризует скорость деформаций, и, соответственно, называется тензором скоростей деформаций. Второе же слагаемое --- тензор антисимметричный, потому что является производной ортогонального тензора при том его значении, когда он сам равняется $I$. Это можно получить дифференцированием по времени тождества $R \cdot R^{T} = I$. Этот антисимметричный тензор мы обозначим $\dot{\omega}\lf t_0 \ri$ и будем называть тензором скоростей поворота. Их сумму, то есть вектор $\dot{F}_{t_0}\lf t \ri\vert_{t = t_0}$, обозначим за $\dot{H}\lf t_0 \ri$ и назовем тензором скоростей дисторсий. Теперь становится видно, что все перечисленные величины относятся к одному и тому же моменту времени $t_0$, который, вообще говоря, совершенно произвольный, поэтому мы могли бы писать здесь не $t_0$, а $t$: $\dot{\epsilon}\lf t \ri $, $\dot{\omega}\lf t \ri$, $\dot{H}\lf t \ri$. \\
	Поскольку мы с вами знаем, что представление произвольного тензора второго ранга в виде суммы симметричного и антисимметричного тензоров единственно. Тогда можно записать $\dot{H} = \dot{\epsilon}+\dot{\omega}$ в любой момент времени и
	\begin{align}
	\dot{\epsilon} &= \dfrac{1}{2}\lf \dot{H} + \dot{H}^T \ri &
	\dot{\omega} = \dfrac{1}{2}\lf \dot{H} - \dot{H}^T \ri.
	\end{align}
	Это не является определением, но, тем не менее, является свойством тензоров $\dot{\epsilon}\lf t \ri $ и $\dot{\omega}\lf t \ri$.\\
	Продолжая формулу \eqref{eq: dr/dt t=t0}, можно записать
	\begin{equation}\label{eq: dr/dt}
	d\dot{\mf{r}}\lf t\ri\vert_{t = t_0} = d\mf{r}\lf t_0\ri \cdot \dot{F}_{t_0}\lf t\ri\vert_{t = t_0} = d\mf{r}\lf t_0 \ri \cdot \dot{H} \lf t_0 \ri 
	\end{equation}
	А так как величины относятся к одному и тому же моменту времени в одной материальной точке, то, отбросив «внутренность», заметим, что все сводится к $d\dot{\mf{r}} = d\mf{r} \cdot \dot{H}$. \\
	Попробуем выразить $\dot{H}$ и, соответственно, $\dot{\epsilon}$ и $\dot{\omega}$ через поле скоростей. Раз это скоростные величины, наверное, они как-то связаны с полем скоростей. Это действительно так, и мы с вами сейчас эту зависимость получим.\\
	В какой-то момент времени $t$
	\begin{equation}
	\dot{H} = F^{-1} \cdot \dot{F}
	\end{equation}
	$\dot{F}$ найдем из соотношения \eqref{eq:grad} путем дифференцирования по времени:
	\begin{equation}
	\lf F\lf\mf{x},\, t\ri \ri^{\displaystyle\cdot} = \lf \nabla_{\kappa}\diad\mf{r}\lf \mf{x},\, t\ri\ri^{\displaystyle\cdot} = \nabla_{\kappa}\diad\dot{\mf{r}}\lf \mf{x},\, t\ri = \nabla_{\kappa}\diad\mf{v}\lf \mf{x},\, t\ri 
	\end{equation}
	Таким образом, используя связь между отсчётным и пространственным градиентом, получаем:
	\begin{equation}\label{eq:distors}
	\dot{H} = \dot{\epsilon} + \dot{\omega} = F^{-1}\cdot \dot{F}  =F^{-1}\cdot\nabla_{\kappa}\diad\mf{v}\lf \mf{x},\, t\ri= \nabla\diad\mf{v} 
	\end{equation}
	Значит, $\dot{H}$ есть градиент $\mf{v}$. Тогда, $\dot{\epsilon}$ --- симметризованный градиент~$\mf{v}$ 
	\begin{equation}
	\dot{\epsilon} = \nabla\stackrel{s}\diad\mf{v}
	\end{equation}
	А $\dot{\omega}$ --- антисимметризованный градиент $\mf{v}$. Теперь можно ввести еще одну величину. Мы знаем, что с любым антисимметричным тензором связан вектор, например вот так:
	\begin{equation}
	\underline{\dot{\omega}} = \nabla\stackrel{a}\diad\mf{v} = \underline{E\cdot\dot{\mf{\phi}}} = \dot{\mf{\phi}}\cdot E 
	\end{equation}
	Этот вектор $\dot{\mf{\phi}}$ называется вектором угловой скорости. Давайте выразим его через $\dot{\omega}$, а потом через $\dot{H}$. Для этого умножим подчеркнутое равенство двойным скалярным произведением слева на альтернирующий тензор $E$
	\begin{equation}
	E:\dot{\omega} = E:E\cdot\dot{\mf{\phi}} = 2I\cdot\dot{\mf{\phi}} = 2\dot{\mf{\phi}}   
	\end{equation}
	Заметим, что выражение $E:\dot{\epsilon} = 0$, так как является произведением антисимметричного и симметричного тензоров. Поэтому, без ограничения общности, в следующее выражение можно добавить $\dot{\epsilon}$
	\begin{equation}
	\dot{\mf{\phi}} = \dfrac{1}{2}E:\dot{\omega} = \dfrac{1}{2} E: \lf\dot{\omega} + \dot{\epsilon}\ri = \dfrac{1}{2} E:\dot{H} = \dfrac{1}{2}E:\nabla\diad\mf{v}
	\end{equation}  
	Продолжая цепочку рассуждений \eqref{eq: dr/dt t=t0} и \eqref{eq: dr/dt}, с учетом того, что все величины относятся к одному и тому же моменту времени, можно записать
	\begin{equation}
	\dot{d\mf{r}} = d\mf{r}\cdot\dot{H} = d\mf{r}\cdot\dot{\epsilon} + d\mf{r} \cdot \dot{\omega} =d\mf{r}\cdot\dot{\epsilon} +d\mf{r}\cdot E\cdot\dot{\mf{\phi}}
	\end{equation}
	Это выражение показывает, что происходит мгновенно с элементом~$d\mf{r}$
	\begin{equation}
	\dot{d\mf{r}} = d\mf{r}\cdot\dot{\epsilon} +\dot{\mf{\phi}} \times d\mf{r}
	\end{equation}
	Скорость изменения элемента $d\mf{r}$ складывается из двух слагаемых. Первое связано с действием тензора скоростей деформаций, а другое связано с вращением, с угловой скоростью $\dot{\mf{\phi}}$. \\
	Как уже говорилось ранее, теория деформаций является линеаризацией теории больших деформаций около отсчётной конфигурации.  А то, что мы делали сейчас, есть не что иное как линеаризация всех возможных соотношений около актуальной конфигурации. Мы получили скоростные характеристики
	\begin{align}
	\dot{H} &= \nabla \diad \mf{v}\\
	\dot{\epsilon} &= \nabla\stackrel{s}\diad\mf{v}\\
	\dot{\omega} &= \nabla\stackrel{a}\diad\mf{v}
	\end{align}
	Если обе части любого из этих выражений мы умножим на $\delta t$, то мы получим уже не скорости, а приращения. Можно ввести тензор $\delta t \cdot\dot{H} = \delta H = \nabla\diad\delta\mf{w}$ "--- тензор малых дисторсий, где $\delta\mf{w}$ "--- вектор малого смещения. И тем же образом  тензор малых деформаций относительной актуальной конфигурации: 
	$$\delta\epsilon= \nabla\stackrel{s}\diad\delta\mf{w},$$ 
	тензор малых поворотов
	$$\delta \omega= \nabla\stackrel{a}\diad\delta\mf{w}.$$
	 Очевидно, что все эти величины, называемые инкрементальными, при $\mf{v} = 0$ обратятся в нуль. Можно записать\\
	\begin{equation}
	\delta\lf d\mf{r}\ri=d\mf{r}\cdot\delta\epsilon + \dot{\mf{\phi}}\times d\mf{r} = d\mf{r}\cdot\delta\epsilon + d\mf{r}\cdot\delta\omega
	\end{equation}
	Пусть $\mf{e}_i$ "--- ОНБ, тогда для симметричного тензора существует спектральное разложение по базису: $\dot{\epsilon} = \dot{\epsilon}_1\, \mf{e}_1 \diad\mf{e}_1 +\dot{\epsilon}_2\, \mf{e}_2 \diad\mf{e}_2 + \dot{\epsilon}_3\, \mf{e}_3 \diad\mf{e}_3 $, где $ \dot{\epsilon}_i$ "--- собственные числа тензора скоростей деформаций, называемые главными скоростями деформации, $ \dot{\mf{e}}_i$ "--- главные оси деформаций. Так же можно записать и для $\delta\epsilon$:
	\begin{equation}
	\delta\epsilon =\sum_{i=1}^3\delta\epsilon_i\mf{e_i}\diad\mf{e_i}
	\end{equation}
	Найдем для начала скорость изменения длины элемента $d\mf{r}$. Для этого продифференцируем по времени квадрат его модуля $\vert d\mf{r}\vert^2 = d\mf{r}\cdot d\mf{r}$
	\begin{multline}
	2\vert d\mf{r}\vert\vert d\mf{r}\vert^{\displaystyle\cdot} = d\dot{\mf{r}}\cdot d\mf{r} + d\mf{r}\cdot\dot{\mf{r}}= \\ = d\mf{r}\cdot\dot{H}\cdot d\mf{r} + d\mf{r}\cdot\dot{H}^T \cdot d\mf{r} = d\mf{r}\cdot\lf\dot{H} + \dot{H}^T\ri\cdot d\mf{r}= \\ = d\mf{r}\cdot 2\dot{\epsilon}\cdot d\mf{r}.
	\end{multline}
	Отсюда относительная скорость удлинения элемента $d\mf{r}$, которая не зависит от длины, а только от направляющего вектора
	\begin{equation}
	\dfrac{\vert d\mf{r}\vert^{\displaystyle\cdot}}{\vert d\mf{r}\vert} = \dfrac{d\mf{r}\cdot 2\dot{\epsilon}\cdot d\mf{r}}{\vert d\mf{r}\vert^2} = \mf{e}\cdot\dot{\epsilon}\cdot\mf{e}\Leftrightarrow\dfrac{\delta\vert d\mf{r}\vert}{\vert d\mf{r}\vert} = \mf{e}\cdot\delta\epsilon\cdot\mf{e}
	\end{equation}
	\begin{equation}
	\dfrac{\vert d\mf{r}\vert^{\displaystyle\cdot}}{\vert d\mf{r}\vert} = \mf{e_i}\cdot\dot{\epsilon}\cdot\mf{e_i} = \dot{\epsilon_i}
	\end{equation}
	Главные оси растяжения это ни что иное как относительные скорости удлинения для переменных, направленных по главным осям. Они испытывают растяжение с такими относительными скоростями $\dot{\epsilon_1}$, $\dot{\epsilon_2}$, $\dot{\epsilon_3}$
	\addtocounter{chapter}{1}
	\section{}
	\newcommand{\ont}{^{\displaystyle\cdot}}
	$\gamma\lf t\ri = 0$, $\gamma\lf t + \tau\ri\neq 0$ .Найдем скорость изменения объема:
	\begin{multline}
	\dfrac{\lf dV\ri\ont}{dV} = \dfrac{dV_{\kappa}\lf\det F\ri\ont}{dV_{\kappa}\lf\det F\ri} =\\= \nabla\cdot\mf{v} = I:\dot{H} = I:\lf \dot{\epsilon} + \dot{\omega}\ri =\\= I:\dot{\epsilon} = \dot{\epsilon}_1 + \dot{\epsilon}_2 + \dot{\epsilon}_3 
	\end{multline}
	А теперь скорость сдвига. С одной стороны
	\begin{multline}
	d\dot{\mf{r}}_1 \cdot d\mf{r}_2 + d\mf{r}_1 \cdot\dot{\mf{r}}_2 = \vert d\dot{\mf{r}}_1\vert\vert d{\mf{r}}_2\vert\sin\gamma +\\+ \vert d{\mf{r}}_1\vert\vert d\dot{\mf{r}}_2\vert\sin\gamma + \vert d{\mf{r}}_1\vert\vert d{\mf{r}}_2\vert\dot{\gamma}\cos\gamma = \vert d{\mf{r}}_1\vert\vert d{\mf{r}}_2\vert \dot{\gamma} 
	\end{multline}
	С другой стороны:
	\begin{multline}
	d\dot{\mf{r}}_1 \cdot d\mf{r}_2 + d\mf{r}_1 \cdot\dot{\mf{r}}_2 = d{\mf{r}}_1\cdot\dot{H} \cdot d\mf{r}_2 + d\mf{r}_1 \cdot{\mf{r}}_2\cdot\dot{H} =\\= d\dot{\mf{r}}_1\cdot\lf \dot{H}  + \dot{H}^T\ri\cdot d\mf{r}_2 
	\end{multline}
	Таким образом:
	\begin{equation}
	\dot{\gamma} = \dfrac{d\dot{\mf{r}}_1\cdot\lf \dot{H}  + \dot{H}^T\ri\cdot d\mf{r}_2 }{\vert d{\mf{r}}_1\vert\vert d{\mf{r}}_2\vert}
	\end{equation}
	\subsection{Уравнения совместности}
	Умножим выражение \eqref{eq:distors} векторно слева на вектор набла:
	\begin{equation}
	\nabla\times\dot{H}\lf\mf{r}\ri \equiv 0
	\end{equation}
	Только такие поля могут быть полями скоростей дисторсии. Основной задачей этой лекции является вывод уравнений совместности и доказательство теоремы Чизара. Но сначала придется доказать некоторые равенства и утверждения.  \\
	
	Для достижении нашей цели в первую очередь на надо вспомнить утверждение прошлого семестра, потому что им мы будем активно пользоваться в этой лекции:
	\begin{equation}\label{eq:gradientisomers}
	\nabla\diad\lf L\lf\mf{r}\ri ^{\lf i_1 i_2 \ldots i_k \ri}\ri = \nabla\diad L^{\lf 1 \lf i_1 + 1 \ri \lf i_2 + 1\ri \ldots\lf i_k + 1 \ri\ri}
	\end{equation}
	Тогда применим формулу \eqref{eq:gradientisomers} к следующим выражениям:
	\begin{align}
	&\nabla\diad\lf\nabla\diad\mf{v}^{T} \ri=\nabla\diad\lf\nabla\diad\mf{v}^{\lf 21\ri} \ri = \nabla\diad\nabla\diad\mf{v}^{\lf 132\ri}\\
	&\nabla\diad\dot{\epsilon} = \dfrac{1}{2}\lf \nabla\diad\nabla\diad\mf{v} + \nabla\diad\nabla\diad\mf{v}^{\lf 132\ri}\ri\\
	&\nabla\diad\dot{\epsilon}^{\lf 132\ri}=\nabla\diad\dot{\epsilon}
	\end{align}
	Последнее выражение является следствием симметрии тензора скоростей деформаций. Тогда разных изомеров будет всего 3: $\nabla\diad\dot{\epsilon}^{\lf 132\ri}=\nabla\diad\dot{\epsilon} $, $\nabla\diad\dot{\epsilon}^{\lf 213\ri}=\nabla\diad\dot{\epsilon}^{\lf 312\ri} $, $\nabla\diad\dot{\epsilon}^{\lf 321\ri}=\nabla\diad\dot{\epsilon}^{\lf 231\ri} $. Аналогично для`$\mf{v}$: $\nabla\diad\mf{v}^{\lf 213\ri}=\nabla\diad\mf{v} $, $\nabla\diad\mf{v}^{\lf 321\ri}=\nabla\diad\mf{v}^{\lf 312\ri} $, $\nabla\diad\mf{v}^{\lf 132\ri}=\nabla\diad\mf{v}^{\lf 231\ri} $.
	Запишем выражения для разных изомеров градиента тензора скоростей деформации:
	\begin{align}
	\nabla\diad\dot{\epsilon} &= \dfrac{1}{2}\lf \nabla\diad\nabla\diad\mf{v} + \nabla\diad\nabla\diad\mf{v}^{\lf 132\ri}\ri\\
	\nabla\diad\dot{\epsilon}^{\lf 213\ri} &= \dfrac{1}{2}\lf \nabla\diad\nabla\diad\mf{v} + \nabla\diad\nabla\diad\mf{v}^{\lf 312\ri}\ri\\
	\nabla\diad\dot{\epsilon}^{\lf 231\ri} &= \dfrac{1}{2}\lf \nabla\diad\nabla\diad\mf{v}^{\lf 231\ri} + \nabla\diad\nabla\diad\mf{v}^{\lf 321\ri}\ri
	\end{align}
	Сложим первую и вторую строчки и вычтем первую:
	\begin{equation}\label{eq:isomersparty}
	\nabla\diad\dot{\epsilon} + \nabla\diad\dot{\epsilon}^{\lf 213\ri} - \nabla\diad\dot{\epsilon}^{\lf 231\ri} = \nabla\diad\nabla\diad\mf{v}
	\end{equation}
	Получается, что если тензор скоростей трансформации тождественный ноль, то его градиент и подавно, но тогда $\nabla\diad\mf{v}\lf\mf{r}\ri = const = \dot{\omega}\lf \mf{r}\ri = \dot{\omega}_0$. Тогда получаем формулу Эйлера, которая пользуется большой популярностью в аналитической механики:
	\begin{equation}
	\mf{v}\lf\mf{r}\ri = \mf{v}_0 + \lf \mf{r} - \mf{r}_0 \ri\cdot\dot{\omega}_0 =\mf{v}_0 + \lf \mf{r} - \mf{r}_0 \ri\cdot E\cdot\dot{\mf{\phi}}_0= \mf{v}_0 + \dot{\mf{\phi}}_0 \times\lf \mf{r} - \mf{r}_0 \ri 
	\end{equation}
	Возьмем еще один градиент от выражения \eqref{eq:isomersparty} и умножим слева и справа двойным скалярным произведением на альтернирующий тензор:
	\begin{multline}
	E:\nabla\diad\nabla\diad\dot{\epsilon}:E + E:\nabla\diad\nabla\diad\dot{\epsilon}^{\lf 1324\ri}:E -\\- E:\nabla\diad\nabla\diad\dot{\epsilon}^{\lf 1342\ri}:E =E:\nabla\diad\nabla\diad\nabla\diad\mf{v}:E
	\end{multline}
	Первое слагаемое и правая часть равенства равно нулю как произведение антисимметричных и симметричного тензора, по свойствам альтернирующего тензора $E =- E^{\lf 213\ri}$, тогда:  
	$$E:\nabla\diad\nabla\diad\dot{\epsilon}^{\lf 1342\ri}:E = E:\nabla\diad\nabla\diad\dot{\epsilon}^{\lf 1324\ri}:E^{\lf 213\ri}.$$
	В итоге получим уравнение совместности:
	\begin{equation}\label{eq:sovmestnosti equation}
	E:\nabla\diad\nabla\diad\dot{\epsilon}^{\lf 1324\ri}:E = 0
	\end{equation}
	В другой форме уравнение запишется следующим образом:
	\begin{equation}
	\nabla\times\lf\nabla\times\dot{\epsilon}\ri^T = 0
	\end{equation}
	Теперь мы готовы сформулировать и доказать теорему Чизара.
	\begin{theorem}[{\textbf{Чизара}}]
		Необходимым и достаточным условием существования поля скоростей является выражение \eqref{eq:sovmestnosti equation}.
	\end{theorem}
	$\vartriangleright$\\
	Попытаемся построить еще одно такое антисимметричное поле, чтобы в сумме их градиент удовлетворял уравнению ротор суммы равен нулю из чего будет следовать по теореме об потенциальности нужное утверждение:
	\begin{align}
	&E:\nabla\diad\lf\nabla\diad\dot{\epsilon}^{\lf 213\ri}:E\ri\\
	&\nabla\times\lf\nabla\diad\dot{\epsilon}^{213}:E\ri = 0
	\end{align}
	\begin{equation}
	\nabla\diad\dot{\epsilon}^{213}:E = \nabla\diad\dot{\mf{\phi }}
	\end{equation}
	\begin{multline}\label{eq:nabla omega dot}
	\nabla\diad\dot{\omega} = \nabla\diad\dot{\mf{\phi}}\cdot E = \nabla\diad\dot{\epsilon}^{213}:\mathbb{E}\cdot E =\\= \nabla\diad\dot{\epsilon}^{213}:\lf\mathbf{1} - \mathbf{1^T} \ri = \nabla\diad\dot{\epsilon}^{213} - \nabla\diad\dot{\epsilon}^{231}
	\end{multline}
	Умножим градиент тензора скоростей дисторсии слева дважды скалярно на альтернирующий тензор:
	\begin{equation}
	E:\lf \nabla\diad\dot{\epsilon} + \nabla\diad\dot{\omega}\ri \stackrel{\eqref{eq:nabla omega dot}}=E:\lf\nabla\diad\dot{\epsilon} + \nabla\diad\dot{\epsilon}^{213} - \nabla\diad\dot{\epsilon}^{231}\ri = 0
	\end{equation}
	Первое и второе слагаемое равны, но противоположны по знаку, а третье равно нулю, как произведение антисимметричного тензора на симметричный. Следовательно, $\exists\,\mf{v}\lf\mf{r}\ri$:
	\begin{equation}
	\dot{\epsilon} + \dot{\omega}=\nabla\diad\mf{v}\lf\mf{r}\ri 
	\end{equation}
	$\mathfrak{ 1}$
\end{document}
